% -*- coding: utf-8; mode: latex; -*-

% +++
% latex = "lualatex"
% +++

%
% FIT2023 向け LaTeX クラスファイル
% https://github.com/trueroad/FITpaper-class
%
% サンプルファイル
%
% Copyright (C) 2018, 2019, 2022, 2023 Masamichi Hosoda.
% All rights reserved.
%

\documentclass{FITpaper}

% 図の貼り込み用
\usepackage{graphicx}

% 最終ページで両カラムの下端を揃える
%\usepackage[balance]{nidanfloat}
\usepackage{flushend}
\def\BibTeX{{\rm B\kern-.05em{\sc i\kern-.025em b}\kern-.08em
    T\kern-.1667em\lower.7ex\hbox{E}\kern-.125emX}}
% 和文タイトル
\jtitle{ブラウザ上でユーザが編集可能な言語パターンマッチシステムの構築 }

% 欧文タイトル
\etitle{Building a user-editable language pattern matching system in the browser}

% 著者数:著者の数だけ c を書く
\authors{cc}

% 和文著者名:著者名間に & を書く
% 所属番号を \affmark でつける
\jauthors{%
  桂 辰弥\affmark{1} &
  竹内 孔一\affmark{1}
}

% 欧文著者名:著者名間に & を書く
\eauthors{%
  Tatsuya Katsura &
  Koichi Takeuchi
}

% 所属
% \affmark でつけた番号毎に指定
\afftext{1}{岡山大学}


\begin{document}

\maketitle
\begin{abstract}
  テキスト中の特定の表現を見つけることは,言語および教育分野において必要となることがある.例えば語学学習において,言語を検索するのに用いるコンコーダンサがある.本研究ではユーザ自身が求める表現を検索ブロックで組み合わせてシステムに投入し,事例を検索できるシステムの開発を行っている.先行研究においてJavaScriptとPythonを利用した基本システムを構築したが,システムの本格利用にはいくつかの課題が残されている.そこで本報告では検索エンジンの中心部分であるPrologデータベースの実装の改良,および、大規模なテキストが扱えるためにデータベースをシステムに導入したので、この改良について報告する。 
\end{abstract}
\section{はじめに}
テキスト中の特定のフレーズや表現を見つけることは,言語および教育分野において必要となることがある.テキストデータから特定のキーワードやフレーズの出現位置や文脈を抽出するためのプログラムとしてコンコーダンサがある.コンコーダンサは語学学習において特定のフレーズや表現の使用例を実際の文脈で把握することで,語彙や文法の理解、単語の使用法や文脈の把握に役立ち,学習者の語彙や表現力の向上に役立つ.
パターンマッチングはテキストの表層で検索を行う正規表現とは異なり,情報を抽出したい文を対象に予め関係する文や文の一部に対応する文構造のパターンを用意し,そのパターンに合致する結果を取得するものである.
有名なコンコーダンサの例として,Sketch Engineがある.Sketch Engine\footnote{https://www.sketchengine.eu/},はクエリ言語としてCQL(Corpus Query Language)\footnote{https://www.sketchengine.eu/documentation/corpus-querying/}が使用されており,コーパス内で正規表現や演算子を組み合わせることでパターンマッチを行うことができる.しかしSketch Engineは,コーパスベースの言語分析や統計的な情報抽出が主な役割であるため,直接的に依存関係解析の直接的な機能を提供していない.

ユーザ自らがこれらを考慮してテキスト中の特定フレーズや表現を抽出するようなシステムを構築することは容易ではない.そこで本研究では意味役割付与システムで解析した結果をユーザ自身が求める表現をあらかじめ用意された検索ブロックで組み合わせてシステムに投入し,事例を検索できるシステムの開発を行っている.先行研究 においてWEBアプリケーションとしてJavaScriptとPythonを利用した基本システムを構築したが,
システムの本格利用にはいくつかの課題が残されている.そこで本報告では検索エンジンの中心部分であるPrologデータベースの実装の改良,および、大規模なテキストが扱えるためにデータベースをシステムに導入したので、この改良について報告する.

\section{提案する言語パターンマッチシステム}
\begin{figure}[htbp]
  \centering
  \includegraphics[scale=0.4]{fig/system_fig.png}
  \caption{システムの構成図}
  \label{fig:sys}
\end{figure}
\subsection{言語パターンマッチシステムの概要}

\begin{table}[htbp]
  \caption{Prologの述語一覧}
  {\small
  \begin{center}
    \begin{tabular}{|l||l|l|}\hline
      Predicates                         &  第2引数   & 第3引数               \\
      \hline 

      chunk(SENTENCE\_ID, 0,\_)           &  根ノード ID& 文節 ID                   \\
      morph(SENTENCE\_ID, \_, \_)      & 文節 ID& 形態素 ID                \\
      main(SENTENCE\_ID, \_, \_)         &文節 ID & 主形態素 \\
      part(SENTENCE\_ID, \_, \_)         &文節 ID& 副形態素\\
      role(SENTENCE\_ID, \_, \_)         &文節 ID& 意味役割  \\
      semantic(SENTENCE\_ID, \_, \_)     &文節 ID & 概念\\
      surf(SENTENCE\_ID, \_, \_)          &ノード ID& ノードの表層\\               
      surfBF(SENTENCE\_ID, \_, \_)       &形態素 ID& 形態素の基本形        \\
      sloc(SENTENCE\_ID, \_, \_)&文節/形態素 ID & 出現位置\\
      pos(SENTENCE\_ID, \_, \_)          &形態素 ID& 品詞\\                                        
      dep(SENTENCE\_ID, \_, \_)        &文節 ID & 文節 ID\\
        \hline
    \end{tabular}
  
    \label{tbl:predicates}
  \end{center}
  }
\end{table}
\section{評価実験}

\subsection{実験内容}
\subsection{結果}
\section{考察と結論}

\acknowledgment{%
  謝辞の文章を\texttt{\textbackslash acknowledgment}で指定します。
  使わなければ謝辞は出力されません。
}

%\bibliographystyle{unsrt}

%\bibliography{all,my-results}


\end{document}